%# -*- coding:utf-8 -*-
\documentclass[10pt,aspectratio=169,mathserif]{beamer}		
%设置为 Beamer 文档类型,设置字体为 10pt,长宽比为16:9,数学字体为 serif 风格

%%%%-----导入宏包-----%%%%
\usepackage{ccnu}			%导入 CCNU 模板宏包
\usepackage{ctex}			%导入 ctex 宏包,添加中文支持
\usepackage{amsmath,amsfonts,amssymb,bm}   %导入数学公式所需宏包
\usepackage{color}			 %字体颜色支持
\usepackage{graphicx,hyperref,url}
\usepackage{algorithm, algorithmic}
\usepackage{cite}
\usepackage{multirow}
\usepackage{metalogo}	% 非必须
%% 上文引用的包可按实际情况自行增删
%%%%%%%%%%%%%%%%%%	


\beamertemplateballitem		%设置 Beamer 主题
\usecolortheme{dove}
%%%%------------------------%%%%%
\catcode`\。=\active         %或者=13
\newcommand{。}{.}				
%将正文中的“。”号转换为“.”。中文标点国家规范建议科技文献中的句号用圆点替代
%%%%%%%%%%%%%%%%%%%%%

%%%%----首页信息设置----%%%%
\title[课程报告]{生物智能算法}
\subtitle{——群聚算法}			
%%%%----标题设置


\author[Swarm Intelligence]{
  小组成员名字 \\\medskip
  {\small \url{fengliangming@zju.edu.cn}} \\
  {\small \url{http://www.zju.edu.cn/}}}
%%%%----个人信息设置
  
\institute[ZJU]{
  计算机科学与技术学院 \\ 
  浙江大学}
%%%%----机构信息

\date[Apr. 01 2018]{
  2018年04月01日}
%%%%----日期信息
  
\begin{document}

\begin{frame}
\titlepage
\end{frame}				%生成标题页

\section{提纲}
\begin{frame}
\frametitle{提纲}
\tableofcontents
\end{frame}				%生成提纲页


\section{SI背景介绍}
\begin{frame}
  \frametitle{简介}
  人工生命:研究具有某些生命基本特征的人工系统。包括两方面的内容:
	 \begin{enumerate}
	    \item{研究如何利用计算技术研究生物现象}
	    \item{研究如何利用生物技术研究计算问题}
	  \end{enumerate}
    我们关注的是第二点。已有很多源于生物现象的计算技巧,例如神经网络和遗传算法。现在讨论另一种生物系统---社会系统:由简单个体组成的群落和环境及个体之间的相互行为。
\end{frame}


\begin{frame}
  \frametitle{Nature’s examples of SI(1)}
    \begin{figure}[htbp]
      \centering
      \includegraphics[width=6cm]{pic/si1.png}
      \caption{Fish schooling}
    \end{figure}
\end{frame}

\begin{frame}
  \frametitle{Nature’s examples of SI(2)}
    \begin{figure}[htbp]
      \centering
      \includegraphics[width=8cm]{pic/si2.png}
      \caption{Birds flocking in V-formation}
    \end{figure}
\end{frame}

\begin{frame}
  \frametitle{Nature’s examples of SI(3)}
    \begin{figure}[htbp]
      \centering
      \includegraphics[width=4cm]{pic/si3.png}
      \caption{Termites’ nest}
    \end{figure}
\end{frame}

\begin{frame}
  \frametitle{Nature’s examples of SI(4)}
    \begin{figure}[htbp]
      \centering
      \includegraphics[width=8cm]{pic/si4.png}
      \caption{Bees’ comb}
    \end{figure}
\end{frame}

\begin{frame}
  \frametitle{Nature’s examples of SI(5)}
    \begin{figure}[htbp]
      \centering
      \includegraphics[width=8cm]{pic/si5.png}
      \caption{Ant chain and Ant wall}
    \end{figure}
\end{frame}


\begin{frame}
  \frametitle{群智能}
  $\qquad$模拟系统利用局部信息从而可以产生不可预测的群行为。我们经常能够看到成群的鸟、鱼或者浮游生物。这些生物的聚集行为有利于它们觅食和逃避捕食者。它们的群落动辄以十、百、千甚至万计,并且经常不存在一个统一的指挥者。它们是如何完成聚集、移动这些功能呢?
  \\
  Millonas提出的群体智能的5个原则:
   \begin{enumerate}
      \item{接近性原则:群体应能够实现简单的时空计算}
      \item{优质性原则:群体能够响应环境要素}
      \item{变化相应原则:群体不应把自己的活动限制在一狭小范围}
      \item{稳定性原则:群体不应每次随环境改变自己的模式}
      \item{适应性原则:群体的模式应在计算代价值得的时候改变}
    \end{enumerate}
\end{frame}

\begin{frame}
  \frametitle{对鸟群行为的模拟}
  $\qquad$对鸟群行为的模拟: Reynolds、Heppner和Grenader提出鸟群行为的 模拟。他们发现,鸟群在行进中会突然同步的改变方向,散开或者聚集等。那么一定有某种潜在的能力或规则保证了这些同步的行为。这些科学 家都认为上述行为是基于不可预知的鸟类社会行为中的群体动态学。 在这些早期的模型中仅仅依赖个体间距的操作,也就是说,这中同步是鸟群中个体之间努力保持 最优的距离的结果。
\end{frame}

\begin{frame}
  \frametitle{对鱼群行为的模拟}
  $\qquad$对鱼群行为的研究:生物社会学家E.O.Wilson对鱼群进行了研究。提出:“至少在理论上,鱼群的个体成员能够受益于群体中其他个体在寻找食物的过程中的发现和以前的经验,这种受益超过了个体之间的竞争所带来的利益消耗,不管任何时候食物资源不可预知的分散。”这说明,同种生物之间信息的社会共享能够带来好处。这是PSO的基础。 
\end{frame}

\section{蚁群算法}
%example

\begin{frame}
  \frametitle{name}
	 \begin{enumerate}
	    \item This just shows the effect of the style
	    \item It is not a Beamer tutorial
	    \item Read the Beamer manual for more help
	    \item Contact me only concerning the style file
	  \end{enumerate}
\end{frame}

\begin{frame}
  \frametitle{群聚算法介绍}

  \begin{itemize}
    \item {编译方式}
	    \begin{itemize}
	    	\item  推荐安装完整版的 TeXLive
	    	\item 使用 \XeLaTeX 编译
	    \end{itemize}
    \item 请参考 \LaTeX 和 Beamer 用户文档 
    
    \item 行内数学公式示例 $\sin^2 \theta + \cos^2 \theta = 1$
    \item {行间数学公式示例 \begin{equation}
	    y_{1}=\int \sin x\, {\rm d}x
    \end{equation}	 }   
    \item 基于“华大绿”颜色 \url{http://www.ccnu.edu.cn/}
  \end{itemize}
\end{frame}


\begin{frame}
  \frametitle{内置环境}
	\begin{block}{Slides with \LaTeX}
	    Beamer offers a lot of functions to create nice slides using \LaTeX.
	  \end{block}
	
	  \begin{block}{The basis}
	    内部使用以下主题
	    \begin{itemize}
	      \item split
	      \item whale
	      \item rounded
	      \item orchid
	    \end{itemize}
	  \end{block}
\end{frame}

\begin{frame}
  \frametitle{带数字列表}
	 \begin{enumerate}
	    \item This just shows the effect of the style
	    \item It is not a Beamer tutorial
	    \item Read the Beamer manual for more help
	    \item Contact me only concerning the style file
	  \end{enumerate}
\end{frame}




\section{鱼群算法}
\begin{frame}
\newcommand{\song}{\setCJKfamilyfont{song}}
\newcommand{\xiaoer}{\fontsize{18pt}{18pt}\selectfont}
	\begin{center}
	{\song\xiaoer\textbf{人工鱼群算法}}
	\end{center}
\end{frame}

\begin{frame}
	\frametitle{目录}
	\begin{itemize}
		\item{算法背景}
		\item{算法简介}
		\item{算法实现}
		\item{算法应用}
		\item{参考文献}
	\end{itemize}
\end{frame}

\begin{frame}
	\frametitle{算法背景}
	\begin{itemize}
		\item{提出者:李晓磊等,2002年}
		\item{提出背景:随着人工智能和人工生命的兴起,出现了一些仿生算法,包括蚁群算法和粒子群算法等。将人工智能的思想应用于问题的寻优,实施一些高级的计算方法。应用动物自治体的模式来定义实体,让他们在问题空间中自主的活动,从而达到解决问题的目的。}
	\end{itemize}
\end{frame}

\begin{frame}
	\frametitle{算法简介}
	\begin{itemize}
		\item{鱼群模式}
			\begin{itemize}
				\item{视觉}
				\item{鱼群行为分析}
				\item{人工鱼}
				\item{问题解决}
			\end{itemize}
		\item{人工鱼模型}
		\item{行为描述}
	\end{itemize}
\end{frame}

\begin{frame}
	\frametitle{鱼群模式——视觉}
	\begin{columns}
	\column{.5\textwidth}
		\begin{figure}
			\centering
			\includegraphics[width=0.9\textwidth]{pic/fish1.pdf}
			\caption{人工鱼的视野和移动步长}
		\end{figure}
	\column{.5\textwidth}
	虚拟人工鱼实体的当前状态为X,Visual为其视野范围,状态$X_v$为其在某时刻视点所在的位置,如果该位置的状态优于当前状态,则考虑向该位置方向前进一步,即到达状态$X_{next}$
	\end{columns}
\end{frame}

\begin{frame}
	\frametitle{鱼群模式——鱼群行为分析}
	\begin{columns}
	\column{.5\textwidth}
		\begin{itemize}
			\item{聚群行为:鱼类进化过程中的一种生存方式,大量或少量的鱼都能聚集成群,进行集体觅食和躲避敌害}
			\item{追尾行为:当某一条鱼或几条鱼发现食物时,它们附近的鱼会尾随其后快速游过来,进而导致更远处的鱼也尾随过来}
		\end{itemize}
	\column{.5\textwidth}
		\begin{itemize}
			\item{随机行为:鱼在水中悠闲的自由游动,基本上是随机的,其视也是为了更大范围地寻觅食物或同伴}
			\item{觅食行为:生物的基本行为,趋向食物的一种活动。一般可以认定它是同过视觉或味觉来感知水中的食物量或浓度来选择趋向}
		\end{itemize}
	\end{columns}
\end{frame}

\begin{frame}
	\frametitle{鱼群模式——人工鱼}
	\begin{columns}
	\column{.6\textwidth}
		\begin{figure}
			\centering
			\includegraphics[width=0.9\textwidth]{pic/fish2.pdf}
			\caption{人工鱼实体}
		\end{figure}
	\column{.4\textwidth}
		人工鱼是真实鱼个体的一个虚拟实体,用来进行问题的分析和说明。\\如图所示,可以将人工鱼看作是一个封装了自身数据星系和一系列行为的一个实体,同过感官来接受环境的刺激信息,并通过控制尾鳍来做出相应的应激活动。\\人工鱼所在的环境主要是问题的解空间和其他人工鱼的状态,它在下一刻的行为取决于目前自身状态和目前环境的状态,并且它还通过自身活动的同时来影响环境,进而影响其他同伴的活动。
	\end{columns}
		
\end{frame}
\begin{frame}
	\frametitle{鱼群模式——问题解决}
	实际问题的解决是通过自治体在自主的活动过程中以某种形式表现出来的。在一片水域中,鱼生存的数目最多的地方一般就是本水域中富含营养物质最多的地方,依据这一特点来模仿鱼群的觅食等行为,从而实现全局寻优,这就是鱼群算法的基本思想。\\在寻优过程中,通常会有两种方式表现出来:
	\begin{itemize}
		\item{一种形式是通过人工鱼最终的分布情况来确定最优解的分布,通常随着寻优解过程的进展,人工鱼往往会聚集在极值点的周围,而且,全局最优的极值点周围能聚集较多的人工鱼}
		\item{另一种形式是在人工鱼的个体状态之中表现出来的,即在寻优的过程中,跟踪记录最优个体的状态,类似于遗传算法等的方式}
	\end{itemize}
\end{frame}
\begin{frame}
	\frametitle{人工鱼模型}
	\begin{columns}
	\column{.35\textwidth}
	\flushleft
		\small{\emph{{class Artificial\underline{\hspace{0.5em}}fish}\\{\{}\\Various:\\{\qquad{float AF\underline{\hspace{0.5em}}X[n];}}\\{\qquad{float AF\underline{\hspace{0.5em}}step;}}\\{\qquad{float AF\underline{\hspace{0.5em}}visual;}}\\{\qquad{float try\underline{\hspace{0.5em}}number}}\\{\qquad{float delta;}}\\{Functions:}\\{\qquad{float AF\underline{\hspace{0.5em}}foodconsistence();}}\\{\qquad{void AF\underline{\hspace{0.5em}}move();}}\\{\qquad{float AF\underline{\hspace{0.5em}}follow();}}\\{\qquad{float AF\underline{\hspace{0.5em}}prey();}}\\{\qquad{float AF\underline{\hspace{0.5em}}swarm();}}\\{\qquad{int AF\underline{\hspace{0.5em}}evaluate();}}\\{\qquad{void AF\underline{\hspace{0.5em}}init();}}\\{\qquad{Artificial\underline{\hspace{0.5em}}fish();}}\\{\qquad{virtual~Artificial\underline{\hspace{0.5em}}fish;}}\\{\};}}}
	\column{.45\textwidth}
	\flushleft
		\small{\emph{ \vspace{0.5ex}\\{//AFs positon}\\//the distance that AF can move for each step\\//the visual distance of AF\\//attempt time in the behavior of prey\\//the condition of jamming\vspace{3ex}\\//the food consistence of AFs current position\\//AF move to the next position\\//the behavior of follow\\//the behavior of prey\\//the behavior of swarm\\//evaluate and select the behavior\\//to initialize the AF\vspace{3ex} }}
	\column{.2\textwidth}
		\tiny{Step:人工鱼移动最大步长\\{$\delta:$拥挤度因子}\\Visual:人工鱼感知距离\\个体状态:$(x_1,x_2,···,x_n)$,其中$x_i$为欲寻优的变量\\Y=f(X)表示当前所在位置食物浓度\\人工鱼个体之间的距离:$d_{i,j}=|X_i-X_j|$}
	\end{columns}
\end{frame}
\begin{frame}
	\frametitle{行为描述}
		\begin{columns}
		
		\column{0.55\textwidth}
		\small{觅食行为}\vspace{3ex}\\
			\flushleft\scriptsize{\emph{float Aritificial\underline{\hspace{0.5em}}fish::AF\underline{\hspace{0.5em}}prey()\\{\{}\\{\qquad for($i=0;i<try\underline{\hspace{0.5em}}number;i++)$}\\{\qquad\{}\\{\qquad\qquad$X_j=X_i+Rand()·Visual;$}\\{\qquad\qquad if $(Y_i<Y_j)$}\\{\qquad\qquad\qquad $X_{i|next}=X_i+Rand()·Step·(X_j-X_i)/\lVert X_j-X_i\rVert;$}\\{\qquad\qquad else}\\{\qquad\qquad\qquad $X_{i|next}=X_i+Rand()·Step;$}\\{\qquad\}}\\{ return $AF\underline{\hspace{0.5em}}foodconsistence(X_{i|next});$} \\{\}} }}
		\column{0.55\textwidth}
		\small{聚群行为}\vspace{3ex}\\
			\flushleft\scriptsize{\emph{float Aritificial\underline{\hspace{0.5em}}fish::AF\underline{\hspace{0.5em}}swarm() \\ {\{} \\ {\qquad $n_f=0;X_c=0$} \\ {\qquad for($j=0;i<friend\underline{\hspace{0.5em}}number;j++)$} \\ {\qquad\qquad $if(d_{i.j}<Visual) \{ n_f++;X_c+=X_j;\}$} \\ {\qquad $X_c=X_c/n_f;$}\\{\qquad $if (Y_c/n_f>\delta Y_i)$}\\{\qquad\qquad $X_{i|next}=X_i+Rand()·Step·(X_c-X_i)/\lVert X_c-X_i\rVert;$}\\{\qquad else}\\{\qquad\qquad AF\underline{\hspace{0.5em}}prey();}\\{ return $AF\underline{\hspace{0.5em}}foodconsistence(X_{i|next});$} \\{\}} }}
		\end{columns}
		
\end{frame}

\begin{frame}
	\frametitle{行为描述}
		\begin{columns}
		\column{0.6\textwidth}
		\small{追尾行为}\vspace{3ex}\\
			\flushleft\scriptsize{\emph{float Aritificial\underline{\hspace{0.5em}}fish::AF\underline{\hspace{0.5em}}follow() \\ {\{} \\ {\qquad $Y_max=-\infty;$} \\ {\qquad for($j=0;i<friend\underline{\hspace{0.5em}}number;j++)$} \\ {\qquad\qquad $if(d_{i,j}<Visual \&\& Y_j>Y_max)$} \\ {\qquad\qquad\qquad $\{ Y_max=Y_j; X_max=X_j; \}$ }\\{\qquad $n_f=0$}\\{\qquad for($j=0;i<friend\underline{\hspace{0.5em}}number;j++)$}\\{\qquad\qquad $if(d_{max,j}<Visual) \{ n_f++;\} $}\\{\qquad $if (Y_max/n_f>\delta Y_i)$}\\{\qquad\qquad $X_{i|next}=X_i+Rand()·Step·(X_max-X_i)/\lVert X_max-X_i\rVert;$}\\{\qquad else}\\{\qquad\qquad AF\underline{\hspace{0.5em}}prey();}\\{ return $AF\underline{\hspace{0.5em}}foodconsistence(X_{i|next});$} \\{\}} }}
		\column{0.4\textwidth}
		\small{随机行为}\vspace{3ex}\\
		\vspace{6ex}随机行为的实现即在视野中随机选择一个状态,然后向该方向移动,实际上就是觅食行为的一个缺省行为。
		\end{columns}
\end{frame}

\begin {frame}
	\frametitle{算法描述}
		\begin{columns}
		\column{0.5\textwidth}
		\begin{figure}
			\centering
			\includegraphics[width=0.8\textwidth]{pic/fish3.pdf}
			\caption{算法示意图}
		\end{figure}
		
		\column{0.5\textwidth}
		\small{每个人工鱼探索它的环境状况(包括目标函数的变化情况和伙伴的变化情况),从而选择一种行为\\最终,人工鱼集结在几个局部极值的周围\\一般情况下,在讨论求极大问题时,拥有较大的$AF\underline{\hspace{0.5em}}foodconsistence$值的人工鱼一般处于值较大的极值域周围\\这有助于获取全局极值域,而值较大的极值区域周围一般能集结较多的人工鱼,这有助于判断并获取全局极值。}
		\end{columns}

\end{frame}

\begin{frame}
	\frametitle{算法实现}
	\begin{algorithm}[H]
	\caption{AFA算法}\label{fish_alg}
	\algsetup{linenosize=\tiny} \scriptsize
		\begin{algorithmic}
			\STATE{$AF\underline{\hspace{0.5em}}init()$}
			\WHILE{the resulr is satisfied}
				\STATE{\textbf{switch}$(AF\underline{\hspace{0.5em}}evaluat())$}	
				\STATE{\textbf{\quad case}\quad value1:}
				\STATE{$\qquad AF\underline{\hspace{0.5em}}follow()$}
				\STATE{\textbf{\quad case}\quad value1:}
				\STATE{\qquad $AF\underline{\hspace{0.5em}}swarm()$}
				\STATE{\quad \textbf{default$:$}}
				\STATE{\qquad $AF\underline{\hspace{0.5em}}prey()$}
				\STATE{\textbf{end switch}}			
				\STATE{$AF\underline{\hspace{0.5em}}move()$}
				\STATE{$get\underline{\hspace{0.5em}}result()$}
			\ENDWHILE
			
		\end{algorithmic}
	\end{algorithm}
\end{frame}
\begin{frame}
	\frametitle{各参数对计算时间的影响}
	\begin{figure}
		\includegraphics[width=0.8\textwidth]{pic/fish6.png}
	\end{figure}
\end{frame}
\begin{frame}
	\begin{columns}
	\column{0.65\textwidth}
	\begin{figure}
		\includegraphics[width=1.0\textwidth]{pic/fish7.png}
	\end{figure}
	\column{0.35\textwidth}
	\begin{figure}
		\flushleft
		\includegraphics[width=1.0\textwidth]{pic/fish8.png}
	\end{figure}
	\end{columns}
\end{frame}
\begin{frame}
	\frametitle{算法特点}
	\begin{itemize}
		\item{算法只需要比较目标函数值,对目标函数的性质要求不高}
		\item{算法对初值的要求不高,初值随机产生或设定为固定值均可以}
		\item{算法对参数设定的要求不高,有较大的容许范围}
		\item{算法具备并行处理的能力,寻优速度较快}
		\item{算法具备全局寻优的能力}
	\end{itemize}
\end{frame}
\begin{frame}
	\frametitle{算法应用——旅行商问题}	
	\small{数据来源:http://www.uni-heidelberg.de/iwr/comopt/software/TSPLIB95/;}
	\begin{columns}
	\column{0.5\textwidth}
		\begin{figure}
			\centering
			\includegraphics[width=0.8\textwidth]{pic/fish4.png}
			\caption{Ulysses 22 城市TSP问题的寻优曲线}
		\end{figure}
	\column{0.5\textwidth}
		\begin{figure}
			\centering
			\includegraphics[width=0.8\textwidth]{pic/fish5.png}
			\caption{Att48 城市TSP问题的寻优曲线}
		\end{figure}
	\end{columns}
	\small{在仿真实验中,为了反映算法对组合优化问题的解决能力,没有针对TSP问题的特点添加任何的启发规则,完全依靠人工鱼群算法对解空间的寻优能力,可见,人工鱼群算法具有较快的收敛速度,但是随着问题规模的扩大,其寻优的精度会有所降低,这是有待改进的地方。}
	
\end{frame}

\begin{frame}
	\frametitle{参考文献}
	\begin{thebibliography}{123} 
	\bibitem{fish_bib1} 李晓磊,邵之江,钱积新.一种基于动物自治体的寻优模式:鱼群算法[J].系统工程理论与实践,2002,22(11):32 -38.
	\bibitem{fish_bib2} 李晓磊.一种新到的智能优化方法———人工鱼群算法[D].杭州:浙江大学,2003.
	\bibitem{fish_bib3}李晓磊,钱积新.基于分解协调的人工鱼群优化算法研究[J].电路与系统学报,2003,8(1) : 1 -6.
	\end{thebibliography}
\end{frame}

\section{蜂群算法}
\begin{frame}
\newcommand{\song}{\setCJKfamilyfont{song}}
\newcommand{\xiaoer}{\fontsize{18pt}{18pt}\selectfont}
	\begin{center}
	{\song\xiaoer\textbf{人工蜂群算法}}
	\end{center}
\end{frame}

\begin{frame}
  \frametitle{ABC算法背景}
	\qquad 人工蜂群算法(Attificial Bee Colony,ABC)是由土耳其学者Karaboga于2005年提出,其基本思想是启发 	 	于蜂	群通过个体分工和信息交流,相互协作完成采蜜任务。与传统的优化方法相比,它的主要优点是不需要了解问    	题的特殊信息,只需要对问题进行优劣比较,通过人工蜂群个体的局部寻优行为,最终在群体中使得全局最优值	 	凸现出来,具有较快的收敛速度。但同时也存在以下问题:ABC算法存在“早熟”的收敛性缺陷,即在接近全局最优解时,已陷入局部最优。

\end{frame}

\begin{frame}
	\frametitle{自然界的蜂群}
	\begin{columns}
	\column{.4\textwidth}
		\begin{itemize}
			\item { 3个基本要素}
				\begin{itemize}
					\item { 蜜源Food Source}
					\item { 被雇佣蜂(引领峰)Employed Foragers}
					\item { 未被雇佣蜂Unemployed Foragers}
						\begin{itemize}
							\item { 跟随蜂 }
							\item { 侦查蜂 }
						\end{itemize}
				\end{itemize}
			\item { 2种基本行为}
				\begin{itemize}
					\item {为蜜源招募蜜蜂Recruit}
					\item {放弃蜜源Abandon}
				\end{itemize}
			 \item { 信息交流}
				\begin{itemize}
					\item {舞蹈区:摇摆舞}
				\end{itemize}
		\end{itemize}
	\column{.6\textwidth}
		\begin{itemize}
			\item[ ] 
				\begin{figure}[htbp]
					\centering
					\includegraphics[scale=0.5]{pic/bee1.png}
				\end{figure}
			\item[ ]
				\begin{figure}[htbp]
					\centering
					\includegraphics[scale=0.5]{pic/bee2.png}
				\end{figure}
			\item[ ] 
				\begin{figure}[htbp]
					\centering
					\includegraphics[scale=0.5]{pic/bee3.png}
				\end{figure}
		\end{itemize}
	\end{columns}
\end{frame}


\begin{frame}
	\frametitle{自然界的蜂群}
	\begin{columns}
	\column{.6\textwidth}
		\begin{itemize}
			\item {引领蜂的搜索行为}
				\begin{itemize}
					\item {在舞蹈区进行蜜源信息分享后,发现自己的蜜源质量并不高,放弃蜜源重新变成侦查蜂寻找新蜜源(图中UF线)}
					\item {在舞蹈区跳摇摆舞招募蜜蜂,此时蜂巢里的非雇佣蜂以一定概率跟随引领蜂回到蜜源进行采蜜(图中EF1线)}
					\item {继续在蜜源处采蜜而不进行招募(图中EF2)}
				\end{itemize}
			\item {非雇佣蜂的搜索行为}
				\begin{itemize}
					\item {以侦查蜂的身份,自发搜索蜂巢附近的蜜源(图中S线)}
					\item {在观察完摇摆舞被雇佣成为跟随蜂,开始搜索对应蜜源附近并采蜜(图中R线)}
				\end{itemize}
		\end{itemize}
	\column{.4\textwidth}
		\begin{figure}[htbp]
			\centering
			\includegraphics[width=6cm]{pic/bee4.jpg}
		\end{figure}
	\end{columns}
\end{frame}


\begin{frame}
	\frametitle{自然界的蜂群}
	\begin{columns}
	\column{.6\textwidth}
		\begin{itemize}
			\item {角色转换}
				\begin{itemize}
					\item {引领蜂用于维持优良解(记录当前局部最优解)。}
					\item {跟随蜂用于提高收敛速度(搜索局部最优解的附近空间)。}
					\item {侦查蜂用于增强摆脱局部最优的能力(重新全局搜索)。}
				\end{itemize}
		\end{itemize}
	\column{.4\textwidth}
		\begin{figure}[htbp]
			\centering
			\includegraphics[width=6cm]{pic/bee5.png}
		\end{figure}
	\end{columns}
\end{frame}

\begin{frame}
	\frametitle{ABC算法模型}
	\begin{columns}
	\column{.6\textwidth}
	\qquad 首先初始化种群,派出侦查蜂搜索蜜源,找到蜜源后转换为引领蜂并评估蜜源的质量,所有引领蜂搜索完毕后回到蜂巢,令适应度高的蜜源对应的引领蜂招募跟随蜂,并对蜜源进行邻域搜索,保留较好的解;令适应度低的蜜源对应的引领蜂重新成为侦查蜂搜索新的蜜源,不断循环输出最优解。
	\column{.4\textwidth}
	\begin{table}[]
	\centering
	\caption{采蜜行为与优化问题的映射}
	\label{采蜜行为与优化问题的映射}
	\begin{tabular}{ccll}
	\cline{1-2}
	\multicolumn{1}{|c|}{\textbf{蜂群采蜜行为}} & \multicolumn{1}{c|}{\textbf{优化问题}} &  &  \\ \cline{1-2}
	\multicolumn{1}{|c|}{蜜源位置}            & \multicolumn{1}{c|}{可行解}            &  &  \\ \cline{1-2}
	\multicolumn{1}{|c|}{蜜源质量}            & \multicolumn{1}{c|}{适应度}            &  &  \\ \cline{1-2}
	\multicolumn{1}{|c|}{采蜜速度}            & \multicolumn{1}{c|}{收敛速度}           &  &  \\ \cline{1-2}
	\multicolumn{1}{|c|}{食物源质量最大值}        & \multicolumn{1}{c|}{最优解}            &  &  \\ \cline{1-2}
	\multicolumn{1}{l}{}                  & \multicolumn{1}{l}{}                &  & 
	\end{tabular}
	\end{table}
	\end{columns}
\end{frame}



\begin{frame}
	\frametitle{1. 蜜源初始化}
	\begin{itemize}
		\item { 设解空间的维度为D,初始蜜源的个数为NP,控制参数limit(局部搜索次数阈值)和最大循环数MaxCycle等。}
		\item { 将蜂群分为引领蜂、侦查蜂、跟随蜂三个类型,且引领蜂和跟随蜂各占蜂群的一半,且其数量等于NP。}
		\item { 根据(1)(2)式在搜索空间随机产生NP个蜜源,并为每一个蜜源分配一个引领蜂。 }
			\begin{equation}
				\vec{X}_{i} = [x_{i1}, x_{i2}, ... x_{iD}]   
			\end{equation}
			\begin{equation}
				x_{id} = L_{d} + rand(0,1) * (U_{d} - L_{d}) 
			\end{equation}
			其中$U_{d}$,$L_{d}$分别表示搜索空间的上限和下限。
	\end{itemize}
\end{frame}

\begin{frame}
	\frametitle{2.搜索更新}
	\begin{itemize}
		\item { 在搜索的开始阶段,引领蜂首先计算蜜源的适应度,如(3)式所示。}
			\begin{equation}
				fit_{i} = \left\{  
             		\begin{array}{lr}  
             		1 / (1 + f_{i}),f_{i} \geq 0 &  \\  
             		1 + abs(f_{i}), otherwise    
             	\end{array}  
           		\right.    
			\end{equation}
			其中$f_{i}$表示解的函数值。
		\item { 再在蜜源i的附近根据(4)式搜索一个新的蜜源 $ V_{i} = [v_{i1},v_{i2},... v_{iD}] $,当新蜜源的适应度fit优于xi时,采用贪婪选择方法用新蜜源替代原来的蜜源,否则保留。}
		\item { 当所有的引领蜂完成贪婪选择后,回到蜂巢舞蹈区进行交流蜜源信息。} 
	\end{itemize}
	\begin{equation}
		v_{id} = x_{id} + \varphi * (x_{id} - x_{jd})    
	\end{equation}
	\qquad 其中$ j \neq i $, 表示在NP个蜜源中随机选择一个不等于i的蜜源;$ \varphi $是[-1,1]均匀分布的随机数。
\end{frame}

\begin{frame}
	\frametitle{3.招募跟随蜂}
	\begin{itemize}
		\item {跟随蜂根据引领蜂分享的蜜源信息,按式(5)的方式计算概率并选择跟随,并采用如2一样的贪婪选择方法在所对应的蜜源附近搜索局部最优解。}
	\end{itemize}
	\begin{equation}
	p_{i} = fit_{i} / \sum_{i=1}^{NP} fit_{i}
	\end{equation}
\end{frame}

\begin{frame}
	\frametitle{4.产生侦查蜂}
	\begin{itemize}
		\item {在搜索过程中,若对蜜源Xi邻域的搜索次数达到limit而未找到更好的蜜源,则该蜜源会被放弃,与之对应的引领蜂会变成侦查蜂,如(6)式所示重新在全局空间随机搜索一个新的蜜源。}
	\end{itemize}
	\begin{equation}
	X_{i}^{t+1} = \left\{  
             		\begin{array}{lr}  
             			L_{d} + rand(0,1) * (U_{d} - L_{d}),t_{i} \geq limit &  \\  
             			X_{i}^{t}, otherwise    
             		\end{array}  
              \right.
	\end{equation}
	\begin{itemize}
		\item {记录当前所有蜜蜂找到的最优蜜源,并跳至第2步,重新迭代直到满足最大迭代次数MaxCycle或者小于优化误差时,输出全局最优解。}
	\end{itemize}
\end{frame}

\begin{frame}
	\frametitle{ABC算法框架}
	\begin{algorithm}[H]
	\caption{ABC}\label{bee_alg}
	\algsetup{linenosize=\tiny} \scriptsize
		\begin{algorithmic}
			\STATE{Initialize the food sources $X_i(i=1,2,3,...,n)$ by Eq.1,\\ 
			the colony count,NP; control parameter,limit; the Max cycle count, MaxCycle;}
			
			\FOR {cycle from 1 to MaxCycle do}
				\FOR {each employee bee i do}
					\STATE {Choose a food source $X_{k}$ in the neighbourhood of $X_{i}$; }
					\STATE {select a jth dimension above all dimension;}
					\STATE {Genernate a food source vi in the neighborhood of $x_{i}$ and $x_{k}$ by Eq.2;}
					\STATE {Apply greedy selection between of $x_{i}$ and $x_{k}$;}
				\ENDFOR
				\FOR {each onlooker bee i do}
					\STATE {select a food source $X_{i}$ depending on probability pi using Eg.3; }
					\STATE {Choose a food source $X_{k}$ in the neighbourhood of $X_{i}$; }
					\STATE {Genernate a food source vi in the neighborhood of $x_{i}$ and $x_{k}$ by Eg.2;}
					\STATE {Apply greedy selection between of $x_{i}$ and $x_{k}$;}
				\ENDFOR
				\IF {there exits an abondoned food source}
					\STATE {Scout bee determines a new food source by Eq.1; }
				\ENDIF
				\STATE {Update best food source;}
			\ENDFOR
		\end{algorithmic}
	\end{algorithm}
\end{frame}

\begin{frame}
	\frametitle{比较}
	\begin{table}[!htbp]
		\centering
		\caption{Numerical benchmark function}
		\label{my-label}
		\begin{tabular}{|c|c|c|cc}
		\cline{1-2}
		Name     		& function  														& range 		&  &  \\ \cline{1-3}
		Sphere函数 		& $f_{1}(x)= \sum_{i=1}^d x_{i}^2 $     							& [-100,100]	&  &\\ \cline{1-3}
		Rosenbrock函数	& $f_{2}(x)= \sum_{i=1}^d (100(x_{i+1}-x_{i}^2)^2 + (x_{i} - 1)^2 ) $ 	& [-30,30] 		&  &  \\ \cline{1-3}
		Griewank函数    	& $f_{3}(x)= \frac{1}{4000} \sum_{i=1}^d x_{i}^2 -  \prod_{i=1}^d cos(\frac{x_{i}}{\sqrt{i}}) + 1 $   	& [-600,600]	&  &  \\ \cline{1-3}
		Rastrigin函数   	& $f_{4}(x)=\sum_{i=1}^d (x_{i}^2 - 10cos(2   \pi 2 x_{i} + 10)$     	& [-5.12,5.12] 	&  &  \\ \cline{1-3}
		\end{tabular}
	\end{table}

	其中函数 $ f_{1}(x) $是单峰函数; 函数 $ f_{2}(x) $ 是很难极小化的病态函数; 函数 $ f_{3}(x) $, $ f_{4}(x) $都是具有大量局部最优点的多峰函数,

\end{frame}

\begin{frame}
	\frametitle{比较}

	\begin{figure}[htbp]
			\centering
			\includegraphics[width=1.0\textwidth]{pic/bee10.png}
			\caption{四种算法运行平均最优值比较}
			\label{fig:pt3}
	\end{figure}

\end{frame}

\begin{frame}
	\frametitle{比较}

		\begin{figure}[htbp]
			\centering
			\includegraphics[width=0.4\textwidth]{pic/bee11.png}
			\caption{四种算法测试最小数随迭代次数的变化过程曲线}
			\label{fig:pt4}
		\end{figure}


\end{frame}

\begin{frame}
	\frametitle{改进}
	\begin{itemize}
		\item {基于混沌鲶鱼效应的人工蜂群算法:一方面通过使用混沌映射的方法替代原来的一般随机化初始蜜源过程;另一方面,将鲶鱼效应施加到蜜蜂种群内,并以混沌波动的方式衍生出一种新型的混沌鲶鱼蜂,用来替代原侦查蜂负责搜索新蜜源的任务,且同原蜂群个体之间形成了有效的竞争、协作机制}
		\item {搜索机制的改进:采用局部随机搜索算子对当前最优蜜源进行局部搜索,以加快搜索速度,同时利用基于排序的选择概率替代原算法中直接利用适应度计算的概率,来保证解的多样性。}
	\end{itemize}
\end{frame}


\begin{frame}
	\frametitle{Application:TSP问题}
	\begin{columns}
	\column{.6\textwidth}
	\qquad 给定N个城市 $C=(C_{1},C_{2}...C_{N})$,求一条从一个城市出发拜访N个所有城市的道路 $(C_{n1},C_{n2}...C_{nN})$,且每个城市有且仅能访问一次,最终回到开始的城市,其中任意两个城市的距离为d(Ci,Cj),使得求得的路径距离最小。
	\column{.4\textwidth}
	\begin{figure}[htbp]
		\flushleft
		\includegraphics[width=3cm]{pic/bee6.jpg}
		\caption{TSP问题}
	\end{figure}
	\end{columns}
\end{frame}


\begin{frame}
	\frametitle{TSP问题}
	\begin{columns}
	\column{.6\textwidth}
	\qquad 所有城市的任一种排列即是问题的一个解,因此初始解空间就是N个城市的排列组合。在人工蜂群算法中,将城市个数N作为解空间的维度,每一个蜜源的位置表示其中一个路径的组合,用这条路径的距离长度表示蜜源的适应度,也就是说,适应度越小的蜜源,所表示的路径也就最优。\\
	\qquad 引领蜂和跟随蜂在更新蜜源位置时,是选择其对应的路径中任意两处进行调换生成新的路径,表示新的位置。
	\column{.4\textwidth}
	\begin{table}[]
	\centering
	\caption{TSP与ABC算法映射}
	\label{my-label}
	\begin{tabular}{|l|l|lll}
	\cline{1-2}
	TSP问题     & ABC算法  &  &  &  \\ \cline{1-2}
	访问所有城市的路径 & 蜜源位置   &  &  &  \\ \cline{1-2}
	路径长度      & 蜜源的适应度 &  &  &  \\ \cline{1-2}	
	最短路径      & 最优蜜源   &  &  &  \\ \cline{1-2}
	\end{tabular}
	\end{table}
	\end{columns}
\end{frame}

\begin{frame}
	\frametitle{算法框架}
	\begin{algorithm}[H]
	\caption{ABC on  TSP}\label{bee_alg}
	\algsetup{linenosize=\tiny} \scriptsize
		\begin{algorithmic}
			\STATE{Initialize the parameter: colony size N, maximum number of iteration MaxCycle, limit.}
			\STATE{Initialize the food sources $x_i(i=1,2,3,...,N)$ and compute the fit value.}
			
			\FOR {cycle from 1 to MaxCycle do}
				\FOR {each employee bee i do}
					\STATE {produce a food source $v_{i}$ in the neighbourhood of $X_{i}$; }
					\STATE {Apply greedy selection between of $x_{i}$ and $v_{i}$;}
				\ENDFOR
				\FOR {each onlooker bee i do}
					\STATE {select a food source $x_{i}$ depending on probability pi ; }
					\STATE {produce a food source $v_{i}$ in the neighbourhood of $X_{i}$; }
					\STATE {Apply greedy selection between of $x_{i}$ and $v_{i}$;}
				\ENDFOR
				\IF {there exits an abondoned food source}
					\STATE {Scout bee determines a new food source. }
				\ENDIF
				\STATE {Update best food source;}
			\ENDFOR
		\end{algorithmic}
	\end{algorithm}
\end{frame}

\begin{frame}
	\frametitle{仿真实验}
	\begin{figure}[htbp]
	\begin{minipage}[t]{4cm}
	\centering  
	\includegraphics[width=4cm]{pic/bee7.png}  
	\caption{初始路径图}
	\end{minipage}  
	\begin{minipage}[t]{4cm}
	\centering  
	\includegraphics[width=4cm]{pic/bee8.png}  
	\caption{结果路径图}
	\end{minipage}  
	\begin{minipage}[t]{4cm}  
	\flushright
	\includegraphics[width=4cm]{pic/bee9.png}  
	\caption{优化曲线}  
	\end{minipage}  
	\end{figure}  
\end{frame}


\begin{frame}
	\frametitle{参考文献}
	\begin{thebibliography}{123456} 
	\bibitem{bee_bib1} 何尧,刘建华,杨荣华.人工蜂群算法研究综述[J/OL].计算机应用研究,2018(05):1-8
	\bibitem{bee_bib2} 陈阿慧,李艳娟,郭继峰.人工蜂群算法综述[J].智能计算机与应用,2014,4(06):20-24
	\bibitem{bee_bib3} Karaboga D. An idea based on honey bee swarm for numerical optimization[R]. Technical report-tr06, Erciyes university, engineering faculty, computer engineering department, 2005.
	\bibitem{bee_bib4} 王慧.人工蜂群算法的性能比较研究[J].河北工程技术高等专科学校学报,2015(01):41-44.
	\bibitem{bee_bib5} 王生生,杨娟娟,柴胜.基于混沌鲶鱼效应的人工蜂群算法及应用[J].电子学报,2014,42(09):1731-1737.
	\bibitem{bee_bib6} 黄秋菀, 王志刚, 夏慧明. 求解旅行商问题的人工蜂群算法 [J]. 价值工程,2013,32(09):206-207
	
	\end{thebibliography}
\end{frame}


\section{狼群算法}
\begin{frame}
\newcommand{\song}{\setCJKfamilyfont{song}}
\newcommand{\xiaoer}{\fontsize{18pt}{18pt}\selectfont}
	\begin{center}
	{\song\xiaoer\textbf{狼群算法}}
	\end{center}
\end{frame}

\begin{frame}
	\frametitle{GWO背景}
	灰狼优化算法(Grey Wolf Optimizer, GWO)是一种模拟灰狼捕食行为的群体智能算法,该算法最先
	由澳大利亚学者 Mirjalili 于 2014 年提出[1],根据灰狼的社会等级将包围、追捕、攻击等捕食任务分配给不
	同等级的灰狼群来完成捕食行为,从而实现全局优化的过程。 GWO 算法具有操作简单、调节参数少、编
	程易实现等特点。在函数优化方面,与其他群智能优化算法相比有明显的优越性。但同时也存在着易陷
	入局部最优、求解精度不高、收敛速度慢等缺点。
\end{frame}


\begin{frame}
	\frametitle{GWO背景}
	\begin{columns}
	\column{.6\textwidth}
		\begin{itemize}
			\item {狼群中的4个等级}
				\begin{itemize}
					\item {$\alpha$: 头狼,狼群的指挥}
					\item {$\beta$: 辅助头狼进行决策}
					\item {$\delta$: 有经验的老狼,警戒/救助}
					\item {$\omega$: 平衡种群内部关系,协助捕猎}
				\end{itemize}
			\item {狼群的3类行为}
				\begin{itemize}
					\item {跟踪, 追逐, 接近猎物}
					\item {追逐, 包围, 骚扰猎物, 直到猎物停止移动}
					\item {向猎物发起攻击}
				\end{itemize}
		\end{itemize}
	\column{.4\textwidth}
		\begin{figure}[htbp]
			\centering
			\includegraphics[width=6cm]{pic/wolf1.png}
			\caption{狼群等级结构}
		\end{figure}
	\end{columns}
\end{frame}


\begin{frame}
	\frametitle{GWO背景}
		\begin{figure}[htbp]
			\centering
			\includegraphics[width=10cm]{pic/wolf2.png}
			\caption{狼群的狩猎行为:A:跟踪猎物;B-D:追逐包围猎物;E:发起攻击}
		\end{figure}
\end{frame}


\begin{frame}
	\frametitle{数学模型——包围/encircling}
	\begin{columns}
	\column{.5\textwidth}
		\begin{equation}
			\vec{D}=|\vec{C} \cdot \vec{X}_{p}(t)-\vec{X}(t)|
		\end{equation}
		\begin{equation}
		\vec{X}(t+1)=\vec{X}(t)-\vec{A}-\vec{D}
		\end{equation}
		其中,$t$代表当前的迭代次数,$\vec{A}$和$\vec{C}$是系数向量,$\vec{X}_p$是当前估计的猎物位置向量,$\vec{X}$是灰狼的位置向量。
		\begin{equation}
			\vec{A}=2\vec{a} \cdot \vec{r}_{1}-\vec{a}
		\end{equation}
		\begin{equation}
		\vec{C}=2 \cdot \vec{r}_2
		\end{equation}
		其中,$\vec{a}$取值在0和2之间,且在迭代过程中逐渐变小,$\vec{r}_1$和$\vec{r}_2$是取值在$[0,1]$之间的随机向量。
	\column{.5\textwidth}
		\begin{figure}[htbp]
			\centering
			\includegraphics[width=7cm]{pic/wolf3.png}
			\caption{3D空间中灰狼的下一个可能位置}
		\end{figure}
	\end{columns}
\end{frame}


\begin{frame}
	\frametitle{数学模型——捕猎/hunting}
	\begin{itemize}
		\item {实际上,猎物位置$\vec{X}_p$是未知的}
		\item {假设把$\alpha$狼的位置作为最佳候选解,$\beta$和$\delta$次之}
		\item {根据这三个较优解,狼群的每个个体开始进行移动}
	\end{itemize}
	\begin{align}
	&\vec{D}_{\alpha}=|\vec{C}_1 \cdot \vec{X}_{\alpha}-\vec{X}|,\quad \vec{D}_{\beta}=|\vec{C}_2 \cdot \vec{X}_{\beta}-\vec{X}|,\quad \vec{D}_{\delta}=|\vec{C}_3 \cdot \vec{X}_{\delta}-\vec{X}| \\
	&\vec{X}_1=\vec{X}_{\alpha}-\vec{A}_1 \cdot \vec{D}_{\alpha},\quad \vec{X}_2=\vec{X}_{\beta}-\vec{A}_2 \cdot \vec{D}_{\beta},\quad \vec{X}_3=\vec{X}_{\delta}-\vec{A}_3 \cdot \vec{D}_{\delta} \\
	&\vec{X}(t+1)=\cfrac{\vec{X}_1+\vec{X}_2+\vec{X}_3}{3}
	\end{align}
\end{frame}

\begin{frame}
	\frametitle{数学模型——捕猎/hunting}
	\begin{figure}[htbp]
		\centering
		\includegraphics[width=10cm]{pic/wolf4.png}
		\caption{狼群hunting行为示意图}
	\end{figure}
\end{frame}

\begin{frame}
	\frametitle{数学模型——攻击搜寻/attacking,exploration}
	\begin{columns}
	\column{.4\textwidth}
		\begin{itemize}
			\item {Attacking prey (exploitation)}
				\begin{itemize}
					\item {当猎物停止移动时,狼群会发起攻击}
					\item {该过程由a的递减实现,A在[-a,+a]之间变化}
					\item {当|A|<1时,狼群的下一个位置会更加接近猎物位置}
				\end{itemize}
			\item {Search for prey (exploration)}
				\begin{itemize}
					\item {当$|A|>1$时,狼群会远离猎物位置}
					\item {$C$可以看做狼群接近猎物的障碍,它影响了狼和猎物之间距离的衡量,这种游走行为会降低陷入局部最优解的可能性}
				\end{itemize}
		\end{itemize}
	\column{.6\textwidth}
		\begin{figure}[htbp]
			\centering
			\includegraphics[width=7cm]{pic/wolf5.png}
			\caption{攻击猎物(a)和搜寻猎物(b)}
		\end{figure}
	\end{columns}
\end{frame}


\begin{frame}
	\frametitle{GWO算法}
	\begin{algorithm}[H]
	\caption{GWO}\label{wolf_alg}
	\algsetup{linenosize=\tiny} \scriptsize
		\begin{algorithmic}
			\STATE{Initialize the grey wolf population $X_i(i=1,2,3,...,n)$}
			\STATE{Initialize a, A and C}
			\STATE{Calculate the fitness of each search agent}
			\STATE{$X_{\alpha}$: the best search agent}
			\STATE{$X_{\beta}$: the second best search agent}
			\STATE{$X_{\delta}$: the third best search agent}
			\WHILE{$t < $ Max number of iterations}
				\FOR{each search agent}
					\STATE{Update the position of current search agent}
				\ENDFOR
				\STATE{Update a, A and C}
				\STATE{Caulate the fitness of all aearch agents}
				\STATE{Update $X_{\alpha}$, $X_{\beta}$ and $X_{\delta}$}
				\STATE{$t=t+1$}
			\ENDWHILE
			\STATE{return $X_{\alpha}$}
		\end{algorithmic}
	\end{algorithm}
\end{frame}

\begin{frame}
	\frametitle{Benchmark-unimodal}
	\begin{table}[]
	\centering
	\caption{unimodal functions}
	\label{wolf_table1}
	\begin{tabular}{lccc}
	\hline
	function & dim & range & $f_{min}$ \\ \hline \hline
	$f_1(x)=\sum_{i=1}^{n}x_i^2$ & 30 &  [-100,100]  & 0  \\ \hline
	$f_2(x)=\sum_{i=1}^{n}|x_i|+\prod_{i=1}^{n}|x_i|$ & 30 & [-10,10] & 0 \\ \hline
	$f_3(x)=\sum_{i=1}^{n}(\sum_{j-1}^{i} x_j)^2$ & 30  & [-100,100] &  0 \\ \hline
	$f_4(x)=max_i\{|x_i|,1 \leq i\leq n\}$ & 30 & [-100,100] &  0 \\ \hline
	$f_5(x)=\sum_{i=1}^{n-1}[100(x_{i+1}-x_i^2)^2+(x_i-1)^2]$ & 30 & [-30,30] &  0 \\ \hline
	\end{tabular}
	\end{table}
\end{frame}


\begin{frame}
	\frametitle{Benchmark-multimodal}
	\begin{table}[]
	\centering
	\caption{multimodal functions}
	\label{wolf_table2}
	\begin{tabular}{p{8cm}lccc}  
	\hline
	function & dim & range & $f_{min}$ \\ \hline \hline
	$f_6(x)=\sum_{i=1}^{n} -x_i \sin (\sqrt{|x_i|})$ & 30 &  [-500,500]  & $-418.9829 \times 5$ \\ \hline
	$f_7(x)=\sum_{i=1}^{n}[x_i^2-10\cos (2 \pi x_i)+10]$ & 30 & [-5.12,5.12] & 0 \\ \hline
	$f_8(x)=-20\exp{\left(-0.2\sqrt{\frac{1}{2}\sum_{i=1}^{n}x_i^2}\right)}- \exp{\left(\frac{1}{n}\sum_{i=1}^{n}\cos(2\pi x_i)\right)+20+e}$ & 30  & [-32,32] &  0 \\ \hline
	$f_9(x)=\frac{1}{4000}\sum_{i=1}{n}x_1^2-\prod_{i=1}^{n}\cos\left(\frac{x_1}{\sqrt{i}} \right)+1 $ & 30 & [-600,600] &  0 \\ \hline
	$f_{10}(x)=\frac{\pi}{n}\{10\sin(\pi y_i)+\sum_{i=1}^{n-1}(y_i-1)^2  [1+10{\sin}^2(\pi y_{i+1})]+(y_n-1)^2\}+\sum_{i=1}^{n}u(x_i,10,100,4) $ & 30 & [-50,50] &  0 \\ \hline
	\end{tabular}
	\end{table}
\end{frame}

\begin{frame}
	\frametitle{Something about PSO}
		\begin{figure}[htbp]
			\centering
			\includegraphics[width=14cm]{pic/wolf6.png}
		\end{figure}
\end{frame}

\begin{frame}
	\frametitle{实验结果}
	\begin{table}[]
	\centering
	\caption{GWO与PSO对比}
	\label{wolf_table3}
	\begin{tabular}{cccccc}
	\multirow{2}{*}{function} & \multirow{2}{*}{min} & \multicolumn{2}{c}{GWO} & \multicolumn{2}{c}{PSO} \\ \cline{3-6} 
	                          &                      & ave        & std        & ave        & std        \\ \hline
	F1                        & 0                    & 6.59e-28   & 6.34e-05   & 1.36e-04   & 2.02e-04   \\
	F2                        & 0                    & 7.18e-17   & 0.029      & 4.21e-02   & 0.045      \\
	F3                        & 0                    & 3.29e-06   & 79.149     & 70.125     & 22.119     \\
	F4                        & 0                    & 5.61e-07   & 1.315      & 1.086      & 0.317      \\
	F5                        & 0                    & 26.81258   & 69.904     & 96.718     & 60.116     \\
	F6                        & -2094.91             & -6123.1    & -4087.44   & -4841.3    & 1152.814   \\
	F7                        & 0                    & 0.310      & 47.356     & 46.704     & 11.629     \\
	F8                        & 0                    & 1.06e-13   & 0.078      & 0.276      & 0.509      \\
	F9                        & 0                    & 0.004      & 0.007      & 0.009      & 0.008      \\
	F10                       & 0                    & 0.053      & 0.020      & 0.007      & 0.026     
	\end{tabular}
	\end{table}
\end{frame}


\begin{frame}
	\frametitle{GWO改进}
	GWO 算法具有操作简单、调节参数少、编程易实现等特点。在函数优化方面,与其他群智能优化算法相比有明显的优越性。但同时也存在着易陷入局部最优、求解精度不高、收敛速度慢等缺点。
	\begin{itemize}
		\item {采用计算分配值的方法提出一种自适应搜索的灰狼求解算法从而加快算法的收敛速度[2]}
		\item {将混沌序列方法引入初始化种群个体,给出了一种寻优性和鲁棒性更好的改进 GWO 算法[3]}
		\item {引入了佳点集理来初始化狼群,并用非固定多段映射罚函数法处理约束条件,利用改进 GWO 算法求解约束优化问题[4]}
	\end{itemize}
	小生境灰狼优化算法(Niche Grey Wolf Optimization, NGWO)[5]利用基本 GWO 算法计算每个灰狼的适应度值。当灰狼距离小于生存半径时,对适应度较差的个体施加惩罚函数,以提高全局搜索能力。
\end{frame}


\begin{frame}
	\frametitle{小生境灰狼算法NGWO}
	\begin{columns}
	\column{.5\textwidth}	
	\begin{itemize}
		\item {“人以类聚,物以群分”}
		\item {将种群划分为多个小生境种群,一小群体为单位进化}
		\item {对小生境种群中适应度较差的个体施加惩罚}
	\end{itemize}
	\column{.5\textwidth}
	个体间距离采用欧氏距离:
	$$d_{ij}=||X_i-X_j||$$
	小生境群体:
	$$d_{ij} < \sigma_{share}$$
	\end{columns}	
\end{frame}


\begin{frame}
	\frametitle{NGWO算法}
	\begin{algorithm}[H]
	\caption{NGWO}\label{nwolf_alg}
	\algsetup{linenosize=\tiny} \scriptsize
		\begin{algorithmic}
			\STATE{Initialize the grey wolf population $X_i(i=1,2,3,...,n)$}
			\STATE{Initialize a, A, C and penatly}
			\STATE{Calculate the fitness $f_i$ of each search agent}
			\STATE{$X_{\alpha}$: the best search agent}
			\STATE{$X_{\beta}$: the second best search agent}
			\STATE{$X_{\delta}$: the third best search agent}
			\WHILE{$t < $ Max number of iterations}
				\FOR{each search agent}
					\STATE{Update the position of current search agent}
				\ENDFOR
				\STATE{Caculate the distance matrix}
				\IF{$d_{ij}<\sigma{share}$}
					\STATE{Set $min\left(f_i,f_j \right)=penalty$}
				\ENDIF
				\STATE{Update a, A and C}
				\STATE{Caulate the fitness of all aearch agents}
				\STATE{Update $X_{\alpha}$, $X_{\beta}$ and $X_{\delta}$}
				\STATE{$t=t+1$}
			\ENDWHILE
			\STATE{return $X_{\alpha}$}
		\end{algorithmic}
	\end{algorithm}
\end{frame}


\begin{frame}
	\frametitle{参考文献}
	\begin{thebibliography}{99} 
	\bibitem{wolf_bib1} Mirjalili, S., Mirjalili, S. and Lewis, A. (2014) Grey Wolf Optimizer. \emph{Advances in Engineering Software}, 69, 46-61
	\bibitem{wolf_bib2} 罗佳, 唐斌. 新型灰狼优化算法在函数优化中的应用[J]. 兰州理工大学学报, 2016, 6(3): 97-101  
	\bibitem{wolf_bib3} 魏政磊, 赵辉, 韩邦杰, 等. 基于自适应 GWO 的多 UCAV 协同攻击目标决策[J]. 计算机工程与应用, 2016, 25(18): 97-101
	\bibitem{wolf_bib4} 龙文, 赵东泉, 等. 求解约束优化问题的改进灰狼优化算法[J]. 计算机应用, 2015, 35(9): 2590-2595
	\bibitem{wolf_bib5} 白媛, 陈京荣, 展之婵. 改进灰狼优化算法的研究与分析[J]. 计算机科学与应用, 2017, 7(6): 562-571
	\end{thebibliography}
\end{frame}

\section{猫群算法}
\begin{frame}
	\frametitle{猫群算法概述}
	猫群算法(Cat Swarm Optimization,缩写为CSO)是由Shu-An Chu等人在2006 年首次提出来的一种基于猫的行为的全局优化算法。根据 生物学分类,猫科动物大约有 32 种,例如: 狮子、老虎 、豹子 、猫等。尽管生存环境不同 ,但是猫科动物的很多生活习性非常相似。猫的警觉性非常高,即使在休息的时候也处于一种高度的警惕状 态,时刻保持对周围环境的警戒搜寻; 它们对于活动的目标具有强烈的好奇心,一旦发现目标便进行跟踪,并且能够迅速地捕获到猎物。猫群算法正是关注了猫的搜寻和跟踪两种行为。
\end{frame}


\begin{frame}
	\frametitle{猫群算法概述}
	猫群算法中,猫即待求优化问题的可行解。猫群算法将猫的行为分为两种模式,一种就是猫在懒散、环顾四周状态时的模式称之为搜寻模式; 另一种是在跟踪动态目标时的状态称之为跟踪模式。猫群算法中,一部分猫执行搜寻模式,剩下的则执行跟踪模式,两种模式通过结合率 MR( Mix- ture Ratio) 进行交互,MR 表示执行跟踪模式下的猫的数量在整个猫群中所占的比例,在程序中 MR 应为一个较小的值。利用猫群算法解决优化问题,首先需要确定参与优化计算的个体数,即猫的数量。每只猫的属性( 包括由 M 维组成的自身位 置) 、每一维的速度、对基准函数的适应值及表示猫是处于搜寻模式或者跟踪模式的标识值。当猫进行完搜寻模式和跟踪模式后,根据适应度函数计算它们的适应度并保留当前群体中最好的解。之后再根据结合率随机地将猫群分为搜寻部分和跟踪部分的猫,以此方法进行迭代计算直到达到预设的迭代次数。
\end{frame}


\begin{frame}
	\frametitle{猫群算法概述}
	\begin{columns}
	\column{.7\textwidth}
		\begin{itemize}
			\item {数学描述}
				\begin{itemize}
					\item {搜寻模式用来模拟猫的当前状态,分别为休息、四处查看、搜寻下一个移动位置。在搜寻模式中,定义了 4 个基本要素: 记忆池(SMP)、变化域 (SRD)、变化数(CDC)、自身位置判断(SPC)。SMP 定义了每一只猫的搜寻记忆大小,表示猫所搜寻到的位置点,猫将根据适应度大小从记忆池中选择一个最好的位置点。SRD表示选择域的变异率,搜寻模式中,每一维的改变范围由变化域决定,根据经验一般取值为 0.2。CDC 指每一只猫将要变异的维数的个数,其值是一个从 0 到总维数之间的随机值。SPC是一个布尔值,表示猫是否将已经过的位置作为将要移动到的候选位置之一,其值不影响 SMP 的取值。}		
				\end{itemize}
		\end{itemize}
	\column{.4\textwidth}
		\begin{figure}[p]
			\centering
			\includegraphics[width=6cm]{pic/cat4.png}
			\caption{流程图}
		\end{figure}
	\end{columns}
\end{frame}

\begin{frame}
	\frametitle{猫群算法概述}
	\begin{columns}
	\column{.6\textwidth}
		\begin{itemize}
			\item {搜寻模式过程描述}
				\begin{itemize}
					\item {将当前位置复制j份副本放在记忆池中,j = SMP,即记忆池的大小为 j; 如果 SPC 的值为真, 令 j = ( SMP - 1) ,将当前位置保留为候选解。}
					\item {对记忆池中的每个个体副本,根据 CDC 的大小,随机地对当前值加上或者减去SRD\%(变化域由百分率表示),并用更新后的值来代替原来的值}
					\item {分别计算记忆池中所有候选解的适应度值。}
					\item {从记忆池中选择适应度值最高的候选点来代替当前猫的位置,完成猫的位置更新。}	
				\end{itemize}
		\end{itemize}
	\column{.4\textwidth}
		\begin{figure}[htbp]
			\centering
			\includegraphics[width=6cm]{pic/cat1.png}
			\caption{搜寻模式}
		\end{figure}
	\end{columns}
\end{frame}

\begin{frame}
	\frametitle{猫群算法概述}
	\begin{columns}
	\column{.7\textwidth}
		\begin{itemize}
			\item {跟踪模式过程描述}
				\begin{itemize}
					\item {速度更新。整个猫群经历过的最好位置, 即目前搜索到的最优解,记做 $X_{best}$ 。每只猫的速度记做$v_i ={v_{i1},v_{i2},...,v_{id}}$,每只猫根据公式(1) 来更新自己的速度。$$v_{i,d}(t+1) = v^{i,d}(t) + r^* c^*(X_{best,d}(t) - x_{i,d}(t)),d = 1,2,…M (1) $$ 
$v_{i,d}(t+1)$表示更新后第 i 只猫在第 d 维的速度值,M 为维数大小; 
$X_{best,d}(t)$ 表示猫群中当前具有最好适应度值的猫的位置; 
$x_{i,d}(t)$ 指当前第 i 只 猫在第 d 维的位置,\\c 是个常量,其值需要根据不同的问题而定。
r 是一个[0,1]之间的随机值。}
					\item {判断每一维的速度变化是否都在SRD内。给每一维的变异加一个限制范围,是为了防止其变化过大,造成算法在解空间的盲目随机搜索。SRD在算法执行之前给定,如果每一维改变后的值超出了SRD的限制范围,则将其设定为给定的边界值。}
				\end{itemize}
		\end{itemize}
	\column{.4\textwidth}
		\begin{figure}[htbp]
			\centering
			\includegraphics[width=6cm]{pic/cat3.png}
			\caption{跟踪模式}
		\end{figure}
	\end{columns}
\end{frame}

\begin{frame}
	\frametitle{猫群算法概述}
	\begin{columns}
	\column{.7\textwidth}
		\begin{itemize}
			\item {跟踪模式过程描述}
				\begin{itemize}
					\item {位置更新。根据公式(2)利用更新后的速度来更新猫的位置。$$x_{i,d}(t+1) = x_{i,d}(t) + v_{i,d}(t+1), (2) d = 1,2,…M$$ 试中$x_i(t+1)$表示第i只猫更新后的位置。}
				\end{itemize}
		\end{itemize}
	\column{.4\textwidth}
		\begin{figure}[htbp]
			\centering
			\includegraphics[width=6cm]{pic/cat3.png}
			\caption{跟踪模式}
		\end{figure}
	\end{columns}
\end{frame}
\begin{frame}
	\frametitle{算法改进研究}
		\begin{itemize}
					\item {Santosa Budi 等( 2009 年)提出一种基于聚类问题的猫群算法,对猫群优化公式进行修正,提高了猫群算法优化聚类问题的优化性能。Yong - Guo Liu 等( 2010 年)引入最新的元启发式方法到猫群算法中,用以寻找最优的数据集聚类方法。}
					\item {引入K - Harmonic 均值操作改善种群并促进 聚类算法的收敛,提出了 2 种基于猫群算法的聚类方法,分别为猫群优化聚类法、K - harmonic 均值猫群优化聚类法。}
					\item {范凯波(2011年)通过研究群体智能计算,提出了基于猫群算法优化的 k-均值聚类算法,实现了车辆目标的分类。}
					\item {Orouskhani Maysam 等(2011年)为提高猫群算法的收敛性,在位置更新方程内增加一个新的参数作为惯性加权,在算法的追踪模型中使用新的速率更新方程,提出一种加权平均惯性猫群算法。}
			
		\end{itemize}

\end{frame}

\begin{frame}
	\frametitle{算法应用研究}
		\begin{itemize}
					\item {王光彪等( 2011 年)针对传统进化算法在图像分类中存在的收敛速度慢、易陷入局部最优等问题,提出用猫群算法求解图像分类问题,将求解组合优化问题转化为猫群的位置寻优过程,并 分析了猫群算法及其两种行为模式下的算法模型。验证了猫群算法在图像分类中的准确性和有效性。}
					\item {Zhi-HuiWang等(2012年)针对最低有效位替换方法解决隐秘图像问题时运行时间长的问 题,通过改进猫群优化策略来获取解决隐秘图像 质量问题的最优解或次优解。}
					\item {Long Xu 等( 2012 年)针对资源受限项目调度问题提出一种基于猫群算法的方法。通过猫的多维位置提供解决资源受限项目调度问题的潜在方案,包括 3 个步骤: 先随机初始化猫的参数,然后迭代位置,通过串行 SGS 方法计算适应度,最后如果条件满足则终止程序。}
					\item {Shi - Yu Cui 等( 2013 年) 针对原始块截断编码方法的复杂性难以找到有效的常用点阵图,利用猫群算法进行块截断编码,提出一种基于此种编码方式的图像压缩技术。}
		
		\end{itemize}

\end{frame}

\begin{frame}
	\frametitle{总结}
	猫群算法的研究刚刚起步,一些思想处于萌芽阶段,严格的理论基础尚不成熟。对于算法本身的思想、原理、参数设置以及种群多样性的研究,仍停留在实验探索阶段,并未有更深入的分析与讨论。关于算法收敛性的分析与证明的研究还未出现。对猫群算法的改进技术主要集中于常态的增加参数、加入部分操作算子等方面,对于算法框架、迭代进化方式等的改进的研究较少。部分学者将粒子群算法以及混沌搜索等算法 或思想引入猫群算法,在一定程度上提高了算法的优化性能,但仍然存在易陷入“早熟”、运行速度 慢等缺陷,并且,被引入算法仅执行猫群算法的单个行为。

\end{frame}

\section{鲸群算法}
\begin{frame}
  \frametitle{whale}
	 \begin{enumerate}
	    \item test
	    \item test
	    \item test
	    \item test
	  \end{enumerate}
\end{frame}

\section{结论}
\begin{frame}
  \frametitle{群智能特点}
  $\qquad$群体中相互合作的个体是分布式的(Distributed),这样更能够适应当前网络环境下的工作状态; 没有中心的控制与数据,这样的系统更具有鲁棒性(Robust),不会由于某一个或者某几个个体的故障而影响整个问题的求解。可以不通过个体之间直接通信而是通过非直接通信(Stimergy)进行合作,这样的系统具有更好的可扩充性(Scalability)。由于系统中个体的增加而增加的系统的通信开销在这里十分小。系统中每个个体的能力十分简单,这样每个个体的执行时间比较短,并且实现也比较简单,具有简单性(Simplicity)。因为具有这些优点,虽说群集智能的研究还处于初级阶段,并且存在许多困难,但是可以预言群集智能的研究代表了以后计算机研究发展的一个重要方向。
\end{frame}


\end{document}